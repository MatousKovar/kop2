\documentclass{article}
\usepackage[utf8]{inputenc}
\usepackage{geometry}
\usepackage{hyperref}
\usepackage{xcolor}
\usepackage{float}
\usepackage{amsmath} 
\usepackage{booktabs}
\usepackage[czech]{babel}
 \geometry{
 a4paper,
 total={170mm,257mm},
 left=20mm,
 top=20mm,
 }
 \usepackage{graphicx}
 \usepackage{titling}

 \title{NI-KOP 2. Úloha}
\author{Matouš Kovář}
\date{Prosinec 2025}
 
 \usepackage{fancyhdr}
\fancypagestyle{plain}{%  the preset of fancyhdr 
    \fancyhf{} % clear all header and footer fields
    \fancyhead[R]{\thedate}
}


\usepackage{titlesec} % for customizing section titles

\titleformat{\section}
  {\normalfont\Large\bfseries\color{cyan}} % or light blue: 'cyan' or custom color
  {\thesection}{1em}{}

\titleformat{\subsection}
  {\normalfont\large\bfseries\color{cyan}}
  {\thesubsection}{1em}{}

\makeatletter
\def\@maketitle{%
  \newpage
  \null
  \vskip 1em%
  \begin{center}%
  \let \footnote \thanks
    {\LARGE \@title \par}%
    \vskip 1em%
    %{\large \@date}%
  \end{center}%
  \par
  \vskip 1em}
\makeatother

\usepackage{lipsum}  
% \usepackage{cmbright}

\begin{document}

\maketitle

\noindent\begin{tabular}{@{}ll}
    \theauthor\\
\end{tabular}

\section{Stručný souhrn}

Tato práce porovnává dva lokální algoritmy pro řešení SAT — \textbf{GSAT} a \textbf{probSAT}. 
Cílem bylo zhodnotit jejich efektivitu podle počtu kroků potřebných k nalezení řešení na instancích s 20, 50 a 75 proměnnými. 

Analýza ukázala, že \textbf{probSAT} systematicky dosahuje lepších výsledků než \textbf{GSAT} z hlediska počtu iterací potřebných k nalezení řešení. \textbf{probSAT} se jeví jako spolehlivá volba solveru pro instance zkoumané v této práci. 


\section{Úvod}

Cílem této semestrální práce je návrh, implementace a experimentální vyhodnocení řešení problému maximální vážené splnitelnosti booleovské formule (MWSAT). Jedná se o optimalizační rozšíření problému SAT, kde je každé klauzuli přiřazena nezáporná váha a úkolem je nalézt takové ohodnocení proměnných, které maximalizuje součet vah splněných klauzulí a zároveň splňuje všechny klauzule. Vzhledem k tomu, že se jedná o problém patřící do třídy NPO, je pro hledání kvalitního řešení v rozumném čase využita pokročilá iterační heuristika – konkrétně \textbf{simulované ochlazování}.\\


Řešení musí být schopno zpracovávat instance rozdílných vlastností a velikostí bez nutnosti interaktivních zásahů do konfigurace. Algoritmus je navrhován tak, aby si poradil s instancemi, které jsou považovány za výpočetně náročné ve smyslu Sellmanova kritéria fázového přechodu.\\

Tento dokument reflektuje postup řešení rozdělený do dvou klíčových fází:

\begin{itemize}
    \item \textbf{White box fáze} -- popisuje proces hledání konfigurace, ladění parametrů a analýzu neúspěšných řešení.
    \item \textbf{Black box fáze} -- předkládá finální experimentální vyhodnocení hotové heuristiky na nezávislé testovací sadě. V této části je ověřena škálovatelnost, stabilita a celková úspěšnost algoritmu.
\end{itemize}

\section{Materiál}

Celá práce je vypracována v programovacím jazyce \textbf{Python}. Pro načítání vstupních dat a ukládání výsledků byly implementovány specializované třídy \texttt{MWSATInstance} a \texttt{MWSATSolution}. Tyto třídy zajišťují nejen manipulaci s daty, ale umožňují i efektivní mapování a sledování postupu algoritmu během výpočtu. Samotný algoritmus je realizován pomocí funkce \texttt{simulated\_annealing}.\\

Pro následnou vizualizaci naměřených dat a tvorbu grafů jsou využity knihovny \textbf{Matplotlib} a \textbf{Seaborn}. Veškeré experimenty byly spouštěny v prostředí Jupyter Notebook na stroji \textbf{MacBook Pro} s procesorem \textbf{M4}, přičemž pro urychlení výpočtů bylo využito 12 procesorových jader.\\

Pro experimentální vyhodnocení byly použity celkem tři velikostní kategorie datových sad. Tyto sady obsahují instance o \textbf{20, 50 a 75 proměnných}. Zatímco sady s 20 a 50 proměnnými obsahují 1000 instancí, sada se 75 proměnnými je omezena na 100 instancí. Klíčovou vlastností použitých dat je jejich variabilita ve váhách: každá velikostní kategorie obsahuje \textbf{čtyři podsady}, které se liší způsobem generování vah jednotlivých klauzulí. Toto rozdělení umožňuje ověřit robustnost heuristiky na instancích s odlišným charakterem váhové funkce.\\


\section{Popis algoritmu a implementace}

Základem implementovaného řešení je stochastická metaheuristika simulovaného ochlazování (Simulated Annealing). Architektura řešení je rozdělena do dvou hlavních tříd: \texttt{MWSATInstance}, která zajišťuje načítání a předzpracování dat, a \texttt{MWSATSolution}, která reprezentuje aktuální stav v prohledávacím prostoru a zapouzdřuje operace nad ohodnocením řešení.

\subsection{Reprezentace dat a vyhodnocovací funkce}
Vstupní data instancí jsou v rámci předzpracování transformována. Klíčovým krokem je \textbf{min-max normalizace vah} proměnných do intervalu $[0, 1]$. Tento krok je zásadní pro robustnost algoritmu, neboť umožňuje definovat hyperparametry (zejména počáteční teplotu a penalizace) nezávisle na absolutní velikosti vah v konkrétní instanci.

Algoritmus využívá hierarchickou vyhodnocovací logiku (lexikografické uspořádání), která striktně upřednostňuje validitu řešení před jeho váhou. Při porovnání dvou stavů $S_{old}$ a $S_{new}$ platí:

\begin{enumerate}
    \item \textbf{Kritérium splnění formule:} Pokud nový splňuje více klauzulí než starý, je považován za striktně lepší bez ohledu na součet vah. Tím je zajištěno směřování do oblasti přípustných řešení.
    \item \textbf{Kritérium maximalizace součtu vah:} Pouze pokud je počet splněných klauzulí shodný, rozhoduje o kvalitě vyšší normalizovaný součet vah aktivních proměnných.
\end{enumerate}

Pro případy kdy nalezneme hoší nové řešení využíváme fitness funkci, na které je závislá pravděpodobnost přijetí tohoto řešení. Tato funkce bude popsána v sekci níže. 

\subsection{Parametry algoritmu}
Chování algoritmu je řízeno sadou hyperparametrů:

\begin{description}
    \item[P0 (Počáteční pravděpodobnost přijetí):] Určuje počáteční teplotu $T_{start}$. Teplota je nastavena tak, aby v úvodní fázi byla průměrná pravděpodobnost přijetí zhoršujícího stavu rovna hodnotě P0.
    
    \item[Cooling Coefficient:] Koeficient ochlazování, kterým je po každém evilibriu násobena aktuální teplota.
    
    \item[Equilibrium Steps:] Nastavuje délku vnitřního cyklu pro danou teplotní hladinu. Parametr je násoben počtem klauzulí, tak aby přirozeně pro instance s více podmínkami prozkoumával déle. 

    
    \item[Fitness Coefficient:] Váha penalizace za nesplněnou klauzuli ve fitness funkci. 
    Musí být nastavena tak, aby algoritmus nikdy neupřednostnil zisk váhy za cenu porušení logické platnosti formule.
    
    \item[Max Steps Without Improvement:] Zastavovací kritérium. Pokud algoritmus nenalezne nové globální maximum po počtu kroků odpovídajícímu 
    $$max\_steps\_without\_improvement * num\_clauses$$
    , tak je ukončen.
    
    \item[Random Flip:] Binární parametr určující strategii výběru proměnné pro změnu stavu (flip):
    \begin{itemize}
        \item \textit{True:} Náhodný výběr jakékoliv proměnné.
        \item \textit{False:} Pokud existují nesplněné klauzule, vybírá se proměnná z náhodně zvolené nesplněné klauzule. To cíleně opravuje validitu řešení.
    \end{itemize}
\end{description}



\subsection{Metodika lazení}
Lazení parametrů bylo provedeno iterativně. Jelikož jsou parametry na sobě závislé, neexistuje nějaký obecný způsob v jakém pořadí by se měly ladit. Každopádně odrazovým můstkem nám bude baseline konfigurace - tedy nějaká konfigurace, která relativně dobře funguje na datech.

Co se týče dat, na kterých budeme parametry ladit, tak zde jsem nejprve používal data s 50 proměnnými a váhamy druhu M. V poslední části evaluace, kde jsem ověřoval funkčnost parametrů na jiných sadách, jsem došel k tomu, že parametry nefungují výrazně tak dobře na sadách R a Q. Proto jsem whitebox fázi udělal znovu, tentokrát ale na datech z instance R. To budou výsledky při volbě parametrů v této práci - \texttt{wuf-50-218-R}.

\subsection{Volba baseline konfigurace}

Jako prvotní konfiguraci pro můj algoritmus jsem zvolil následující tabulku\ref{tab:baseline_params}:

\begin{table}[h!]
\centering
\caption{Výchozí konfigurace parametrů (Baseline)}
\label{tab:baseline_params}
\begin{tabular}{|l|c|l|}
\hline
\textbf{Parametr (kód)} & \textbf{Hodnota} & \textbf{Význam} \\ \hline
\texttt{P0} & 0.9 & Počáteční pravděpodobnost přijetí horšího stavu \\ \hline
\texttt{cooling\_coefficient} & 0.9 & Faktor geometrického ochlazování teploty \\ \hline
\texttt{equilibrium\_steps} & 3 & Násobek $N_{vars}$ pro délku vnitřního cyklu \\ \hline
\texttt{fitness\_coefficient} & 300 & Váha penalizace za nesplněné klauzule \\ \hline
\texttt{max\_steps\_without\_improvement} & 100 & Násobek $N_{clauses}$ pro zastavení při stagnaci \\ \hline
\texttt{random\_flip} & True & Strategie čistě náhodného výběru proměnné \\ \hline
\end{tabular}
\end{table}

Jak si vedla tato konfigurace na 10 bězích 10 různých instancí je možné vidět na tabulce \ref{tab:results_baseline}. Hlavní pozornost by měla být na počtu vyřešených instancí, který je velmi malý. První otázka se tedy rovnou nabízí: Jakvíce tlačit na splnění formule? 

\begin{table}[h!]
\centering
\caption{Výsledky experimentu - Baseline (úspěšnost z 10 běhů)}
\label{tab:results_baseline}
\begin{tabular}{lrr}
\toprule
\textbf{Instance} & \textbf{Solved} & \textbf{Optimal} \\ 
\midrule
wuf50-0323 & 6 & 2 \\
wuf50-0366 & 5 & 5 \\
wuf50-069  & 5 & 5 \\
wuf50-094  & 0 & 0 \\
wuf50-0861 & 2 & 2 \\
wuf50-0824 & 3 & 2 \\
wuf50-0657 & 10 & 3 \\
wuf50-0286 & 0 & 0 \\
wuf50-0612 & 8 & 7 \\
wuf50-0939 & 3 & 2 \\
\bottomrule
\end{tabular}
\end{table}
\subsection{Nastavení výběru proměnné}

Parametr \texttt{random\_flip} nastavený na False zajišťuje, že budou obraceny jen ty proměnné, které jsou obsaženy v nesplňených klauzulích. Tento krok se ukázal jako výhodný, což potvrzuje následující tabulka \ref{tab:results_improved}.


\begin{table}[H]
\centering
\caption{Výsledky vylepšené konfigurace (úspěšnost z 10 běhů)}
\label{tab:results_improved}
\begin{tabular}{lrr}
\toprule
\textbf{Instance} & \textbf{Solved} & \textbf{Optimal} \\ 
\midrule
wuf50-0323 & 10 & 10 \\
wuf50-0366 & 10 & 10 \\
wuf50-069  & 10 & 10 \\
wuf50-094  & 7  & 5 \\
wuf50-0861 & 10 & 9 \\
wuf50-0824 & 9  & 4 \\
wuf50-0657 & 10 & 6 \\
wuf50-0286 & 10 & 10 \\
wuf50-0612 & 10 & 10 \\
wuf50-0939 & 10 & 6 \\
\bottomrule
\end{tabular}
\end{table}

\subsection{Nastavení fitness funkce}

Nyní přejdeme k dalšímu klíčovému parametru a to fitness funkci. Fitness funkce přímo ovlivňuje pravděpodobnost s kterou je horší stav přijat při dané teplotě. Původně byla navržena fitness funkce následující: 
$$
f(s) = \text{Weight}_{norm}(s) - (\text{fintess\_coefficient} \cdot N_{unsat}(s))
$$

kde $\text{Weight}_{norm}$ je součet normalizovaných vah splněných proměnných, $N_{unsat}$ je počet nesplněných klauzulí a \texttt{fitness\_coefficient} je koeficient penalizace ovlivňující jak moc má vliv každá nesplněná klausule.

Tato fitness funkce fungovala poměrně dobře, nicméně při revizi kódu mě napadl hlavní nedostateka to je absence jakékoliv škálovatelnosti na počet proměnných. Například pro instance s větším počtem proměnných bude první člen $\text{Weight}_{norm}$ přirozeně větší zase naopak druhý člen bude pro instance s větším počtem klausulí nabývat vyšší absolutní hodnoty, jelikož může obsahovat větší počet nesplněných klauzulí. 

Proto byla fitness funkce přepracována na následující finální podobu, která adresuje tyto nedostatky: 

$$
f(s) = \text{Weight}_{norm}(s) - (\text{fintess\_coefficient} \cdot \text{num\_variables} \cdot \text{unsat\_ratio})
$$

kde \texttt{num\_variables} je počet proměnných a \texttt{unsat\_ratio} je poměr nesplněných klauzulí vůči všem klauzulím.

Hodnota \texttt{fitness\_coefficient} byla experimentálně nastavena na hodnotu 100 (viz. tabulka \ref{tab:fitness_tuning} a funkce \texttt{run_tuning_experiment}, která byla spouštěna na různých instancích). Na grafu je možné vidět běhy na jedné instanci s růžnými hodnotami. Pro nízké hodnoty je  vidět jak algoritmus upřednostní řešení s větší váhou, protože nejsou dostatečně penalizovány nesplněné klauzule. Solved znamená, že nalezl validní řešení, ale neznamená to, že toto řešení je optimální. Hodnota optima je zvýrazněna červenou čárkovanou čarou. 

\begin{table}[H]
\centering
\caption{Vliv nastavení fitness koeficientu na kvalitu řešení (40 běhů)}
\label{tab:fitness_tuning}
\begin{tabular}{lrrrr}
\toprule
\textbf{Fitness koef.} & \textbf{Solved} & \textbf{Optimal} & \textbf{Prům. \% Opt.} & \textbf{Prům. kroků} \\ 
\midrule
10  & 32 & 16 & 90,30 & 78\,136,6 \\
20  & 36 & 24 & 92,12 & 79\,624,5 \\
50  & 39 & 34 & 97,57 & 80\,376,6 \\
100 & 40 & 30 & 96,31 & 79\,395,6 \\
150 & 38 & 30 & 95,58 & 80\,164,1 \\
200 & 39 & 31 & 97,41 & 79\,461,0 \\
\bottomrule
\end{tabular}
\end{table}

\begin{figure}[H]
    \centering
    % width nastavuje šířku na 80% šířky textu, můžete upravit (např. 1.0\textwidth)
    \includegraphics[width=\textwidth]{fitness_coeficient.png}
    \caption{Vliv parametru fitness coefficient na konvergenci}
    \label{fig:fitness_coeff_10}
\end{figure}


\subsection{Inicializace teploty}
Počáteční teplota je volena na základě vzorce z přednášky: 
$$
T_0 = \frac{\delta}{\lvert\ln P_0\rvert}
$$
kde $\delta$ je průměrná velikost lokálních optim. V praxi jí vypočítávám tak, že spustím algoritmus na 3000 kroků s vysokou teplotou a nízkým \texttt{cooling_coefficient} --- takže to v praxi odpovídá náhodně procházce po stavovém prostoru. V každém kroku odečtu fitness aktuálního a sousedního stavu a absolutní hodnoty zprůměruji. $P_0$ je pravděpodobnost úniku z tohoto minima. 

Parametr, který je lazen je tedy v tomto případě $P_0$. Vyšší hodnota odpovídá vyšší počáteční teplotě --- vyšší pravděpodobnost přijetí zhoršení.

Tuto metodu automatického nastavení teploty jsem zvolil z důvodu, že teplotu automaticky škáluje na různé velikosti instance co do rozdílu mezi váhami a různé hloubky lokálních optim optim. 

Hodnota byla nastavena na \texttt{P0 = 0.8}. Na grafu \ref{fig:p0_graph} je vidět jaký vliv mají 2 různé hodnoty tohoto parametru na 3 bězích na stejné instanci. Na tabulce \ref{tab:p0_tuning}

\begin{figure}[H]
    \centering
    % width nastavuje šířku na 80% šířky textu, můžete upravit (např. 1.0\textwidth)
    \includegraphics[width=\textwidth]{P0_graph.png}
    \caption{Vliv parametru fitness coefficient na konvergenci}
    \label{fig:p0_graph}
\end{figure}


\begin{table}[H]
\centering
\caption{Vliv počáteční pravděpodobnosti $P_0$ na kvalitu řešení}
\label{tab:p0_tuning}
\begin{tabular}{lrrrr}
\toprule
\textbf{$P_0$} & \textbf{Solved} & \textbf{Optimal} & \textbf{Prům. \% Opt.} & \textbf{Prům. kroků} \\ 
\midrule
0,60 & 38 & 21 & 93,87 & 70\,550,2 \\
0,70 & 38 & 22 & 93,42 & 72\,446,9 \\
0,80 & 38 & 25 & 93,93 & 75\,422,6 \\
0,85 & 36 & 27 & 96,60 & 76\,959,4 \\
0,90 & 36 & 27 & 96,12 & 78\,659,9 \\
\bottomrule
\end{tabular}
\end{table}

Hodnota 0.8 se mi zdála nejvhodnější, z důvodu počtu vyřešených a počtu optimálně vyřešených instancí.



\subsection{Nastavení ekvilibria a chladícího koeficientu}

Tyto dva parametry jsou spolu úzce spjaty, proto bude nejvhodnější hledat optimální hodnotu obou parametrů najednou. Výběr jsem prováděl pomocí výběru všech možných kombinací mnou vybraných hodnot parametrů a spustil je 3 krát na 10 různých instancích. Výsledky je možné vidět v tabulce \ref{tab:grid_search_results}


\begin{table}[H]
\centering
\caption{Výsledky grid search pro parametry Cooling a Equilibrium Steps (průměr z 10 instancí)}
\label{tab:grid_search_results}
\begin{tabular}{lrrrr}
\toprule
\textbf{Cooling} & \textbf{Steps} & \textbf{Úspěšnost} & \textbf{Prům. skóre} & \textbf{Prům. kroků} \\ 
\midrule
0,950 & 1 & 86,7 \% & 18\,243,6 & 68\,837 \\
0,950 & 3 & 96,7 \% & 20\,372,1 & 84\,845 \\
0,950 & 4 & 100,0 \% & 20\,487,6 & 87\,170 \\
0,950 & 6 & 100,0 \% & 20\,520,7 & 94\,481 \\
\midrule
0,990 & 1 & 100,0 \% & 20\,475,6 & 93\,732 \\
0,990 & 3 & 100,0 \% & 20\,427,4 & 109\,043 \\
0,990 & 4 & 96,7 \% & 20\,338,8 & 109\,232 \\
0,990 & 6 & 100,0 \% & 19\,730,7 & 103\,724 \\
\midrule
0,995 & 1 & 100,0 \% & 20\,520,7 & 105\,897 \\
0,995 & 3 & 100,0 \% & 19\,695,7 & 105\,010 \\
0,995 & 4 & 100,0 \% & 20\,080,8 & 112\,458 \\
0,995 & 6 & 100,0 \% & 19\,525,1 & 117\,938 \\
\midrule
0,999 & 1 & 100,0 \% & 19\,306,9 & 112\,044 \\
0,999 & 3 & 100,0 \% & 19\,123,1 & 105\,926 \\
0,999 & 4 & 96,7 \% & 19\,368,6 & 106\,965 \\
0,999 & 6 & 100,0 \% & 19\,381,1 & 105\,773 \\
\bottomrule
\end{tabular}
\end{table}

Na základě analýzy naměřených dat lze formulovat následující závěry ohledně vlivu parametrů na chování algoritmu:

\begin{itemize}
    \item \textbf{Kvalita řešení:} Nejvyššího průměrného skóre ($20\,520,7$) dosáhly dvě konfigurace: kombinace koeficientu chlazení $\alpha = 0,995$ s délkou vnitřního cyklu (equilibrium steps) $L=1$ a kombinace $\alpha = 0,95$ s $L=6$. Tyto sady parametrů se jeví jako globálně optimální pro testovanou množinu instancí.
    
    \item \textbf{Efektivita konvergence:} Při srovnání obou nejúspěšnějších konfigurací vykazuje varianta $\alpha = 0,95$ / $L=6$ vyšší výpočetní efektivitu. Ačkoliv obě nastavení vedou ke stejné kvalitě řešení, tato varianta konverguje průměrně v $94\,000$ krocích, zatímco varianta s $\alpha = 0,995$ vyžaduje přibližně $106\,000$ kroků. Rychlejší chlazení kompenzované delším prohledáváním na dané teplotě se tak ukazuje jako výhodnější strategie.
    
    \item \textbf{Vliv limitu iterací:} U extrémně pomalého chlazení ($\alpha = 0,999$) dochází k paradoxnímu poklesu průměrného skóre na hodnotu cca $19\,300$. Tento jev je způsoben tím, že algoritmus v rámci pevně stanoveného limitu maximálního počtu kroků nestihne dostatečně snížit teplotu a zkonvergovat do lokálního optima (tzv. \textit{freezing} není dokončen před vypršením \texttt{max\_steps}).
\end{itemize}


\subsection{Nastavení počtu kroků algoritmu bez zlepšení}

Poslední parametr, který je potřeba nastavit, je parametr \texttt{max_steps_without_improvement}. Tento parametr je násoben počtem klauzulí a tím je získán počet kroků algoritmu který lze provést bez nalezení globálního maxima před ukončením algoritmu. Je potřeba 

Graf 3 běhů na instanci pro hodnoty parametru 300 a 800 je na \ref{fig:max_steps_param}. Kvantitativní výsledky lze vidět na tabulce \ref{tab:results_max_steps}

\begin{figure}[H]
    \centering
    % width nastavuje šířku na 80% šířky textu, můžete upravit (např. 1.0\textwidth)
    \includegraphics[width=\textwidth]{max_steps_improvement_graph.png}
    \caption{Vliv parametru fitness coefficient na konvergenci}
    \label{fig:max_steps_param}
\end{figure}

\begin{table}[h!]
\centering
\caption{Porovnání výsledků pro různé hodnoty parametru max\_steps\_without\_improvement}
\label{tab:results_modified}
\begin{tabular}{rccccc}
\toprule
\textbf{Max Steps} & \textbf{Runs} & \textbf{Solved} & \textbf{Optimal} & \textbf{Avg \% Opt} & \textbf{Avg Steps} \\
\midrule
200  & 40 & 40 & 35 & 98.40  & 72\,888.3 \\
300  & 40 & 40 & 37 & 99.15  & 94\,928.1 \\
600  & 40 & 40 & 39 & 99.85  & 158\,268.0 \\
\textbf{800}  & \textbf{40} & \textbf{40} & \textbf{40} & \textbf{100.00} & \textbf{205\,552.2} \\
1200 & 40 & 40 & 40 & 100.00 & 290\,735.7 \\
\bottomrule
\end{tabular}
\end{table}

Na základě výsledků shrnutých v Tabulce~\ref{tab:results} byla jako finální hodnota parametru \texttt{max\_steps\_without\_improvement} zvolena hodnota 600. Tato konfigurace představuje ideální kompromis mezi spolehlivostí algoritmu a výpočetní náročností.

Důvody pro tento výběr jsou následující:
\begin{itemize}
    \item \textbf{Garance optimality:} Hodnota 800 je nejnižší testovanou konfigurací, která dosáhla 100\% úspěšnosti v nalezení optimálního řešení (40/40). Nižší nastavení, včetně hodnoty 600, vykazovala v několika bězích konvergenci pouze k lokálním optimům.
    \item \textbf{Stabilita vs. Rychlost:} Ačkoliv je nastavení 600 výpočetně méně náročné, riziko nenalezení globálního optima (úspěšnost 39/40) je pro účely této aplikace nepřijatelné. Hodnota 800 tak představuje nutnou investici výpočetního času pro zajištění maximální spolehlivosti.
    \item \textbf{Saturace výkonu:} Další navýšení parametru na 1200 sice udržuje 100\% úspěšnost, ale nepřináší žádnou kvalitativní výhodu oproti hodnotě 800, přičemž neúměrně zvyšuje výpočetní čas (\textit{Avg Steps}). Bod 800 tedy představuje hranici, kde se kvalita řešení stabilizuje.
\end{itemize}


Tímto jsme dokončili volbu parametrů. Finální parametry lze vidět na tabulce \ref{tab:final_params}.

\begin{table}[h!]
\centering
\caption{Finální kombinace parametrů}
\label{tab:final_params}
\begin{tabular}{|l|c|}
\hline
\textbf{Parametr (kód)} & \textbf{Hodnota} \\ \hline
\texttt{P0} & 0.8 \\ \hline
\texttt{cooling\_coefficient} & 0.95 \\ \hline
\texttt{equilibrium\_steps} & 8 \\ \hline
\texttt{fitness\_coefficient} & 100 \\ \hline
\texttt{max\_steps\_without\_improvement} & 800 \\ \hline
\texttt{random\_flip} & False \\ \hline
\end{tabular}
\end{table}

\subsection{Adaptace parametrů na jiné instance}


Poslední částí whitebox fáze bude rozhodnutí, zda parametry fungují i pro jinak velké instance, popřípadě na instance s olišně generovanými vahami.

\begin{figure}[H]
    \centering
    % width nastavuje šířku na 80% šířky textu, můžete upravit (např. 1.0\textwidth)
    \includegraphics[width=\textwidth]{50_m_test.png}
    \caption{Vliv parametru fitness coefficient na konvergenci}
    \label{fig:50_m_test}
\end{figure}

\begin{figure}[H]
    \centering
    % width nastavuje šířku na 80% šířky textu, můžete upravit (např. 1.0\textwidth)
    \includegraphics[width=\textwidth]{50_q_test.png}
    \caption{Vliv parametru fitness coefficient na konvergenci}
    \label{fig:50_q_test}
\end{figure}

\begin{figure}[H]
    \centering
    % width nastavuje šířku na 80% šířky textu, můžete upravit (např. 1.0\textwidth)
    \includegraphics[width=\textwidth]{75_vars_test.png}
    \caption{Vliv parametru fitness coefficient na konvergenci}
    \label{fig:75_vars_test}
\end{figure}



\end{document}


